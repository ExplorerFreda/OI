\documentclass[10pt]{beamer}
\usepackage{times}
\usepackage[T1]{fontenc}
\usepackage{newtxmath,newtxtext}
\usepackage{amsmath}
\usepackage{algorithm}  
\usepackage{algorithmic} 
\usepackage{caption}
\usepackage{animate} 
\usepackage{lipsum} % just to generate dummy text

\usetheme[
%%% option passed to the outer theme
%    progressstyle=fixedCircCnt,   % fixedCircCnt, movingCircCnt (moving is deault)
  ]{Feather}
  
% If you want to change the colors of the various elements in the theme, edit and uncomment the following lines

% Change the bar colors:
%\setbeamercolor{Feather}{fg=red!20,bg=red}

% Change the color of the structural elements:
%\setbeamercolor{structure}{fg=red}

% Change the frame title text color:
%\setbeamercolor{frametitle}{fg=blue}

% Change the normal text color background:
%\setbeamercolor{normal text}{fg=black,bg=gray!10}

%-------------------------------------------------------
% INCLUDE PACKAGES
%-------------------------------------------------------

\usepackage[utf8]{inputenc}
\usepackage[english]{babel}
\usepackage[T1]{fontenc}
\usepackage{helvet}

%-------------------------------------------------------
% DEFFINING AND REDEFINING COMMANDS
%-------------------------------------------------------

% colored hyperlinks
\newcommand{\chref}[2]{
  \href{#1}{{\usebeamercolor[bg]{Feather}#2}}
}

%-------------------------------------------------------
% INFORMATION IN THE TITLE PAGE
%-------------------------------------------------------

\title[Mathematics in OI] % [] is optional - is placed on the bottom of the sidebar on every slide
{ % is placed on the title page
      \textbf{Mathematics in Olympiad in Informatics}
}


\author[Haoyue Shi]
{      Frederica Haoyue Shi \\
      {\ttfamily hyshi@pku.edu.cn}
}

\institute[School of EECS, Peking University]
{
      School of Electronics Engineering and Computer Science\\
	  Peking University \\
  
  %there must be an empty line above this line - otherwise some unwanted space is added between the university and the country (I do not know why;( )
}

\date{\today}

%-------------------------------------------------------
% THE BODY OF THE PRESENTATION
%-------------------------------------------------------

\begin{document}

%-------------------------------------------------------
% THE TITLEPAGE
%-------------------------------------------------------

{\1% % this is the name of the PDF file for the background
\begin{frame}[plain,noframenumbering] % the plain option removes the header from the title page, noframenumbering removes the numbering of this frame only
  \titlepage % call the title page information from above
\end{frame}}


\begin{frame}{Content}{}
\tableofcontents
\end{frame}


%-------------------------------------------------------
\section{Introduction}
%-------------------------------------------------------
\begin{frame}{Introduction}
%-------------------------------------------------------
English is \textbf{very helpful} to our future study and work. Please try to use English as much as possible, from now on. 
\\~\\
Therefore, the slides are written in English. 
\\~\\
If there's no specific instruction, the complexity is always time complexity. 
\\~\\
For the convenience of view, I recommend you to open it with Adobe Reader. 
\end{frame}


%-------------------------------------------------------
\section{Number Theory}
\subsection{Division, Prime and Coprime}
%-------------------------------------------------------
\begin{frame}{Number Theory}{Division, Prime and Coprime}
%-------------------------------------------------------
\begin{block}{Fundamental theorem of arithmetic}
    Every integer greater than $1$ either is prime itself or is the product of prime numbers, and that this product is unique, up to the order of the factors. i.e. $$\forall n > 1, n \in \mathbb{N}, n = p_1^{a_1}p_2^{a_2} ... p_n^{a_n}$$ where $p_k (1\leq k\leq n), p_1<p_2<...<p_n$ is prime. 
\end{block}
\pause
\begin{block}{Prime Number Theorem}
    \begin{equation}
		\pi(N) \sim	 \frac{N}{\log(N)} \nonumber
	\end{equation}
	where $\pi(N)$ is the prime counting function, and $\log(N)$ is the natural logarithm of $N$. The equation means that for large enough $N$, the probability that a random natural number not greater than $N$ is prime is very close to $\frac{1}{\log(N)}$.
\end{block}
\end{frame}


%-------------------------------------------------------
\begin{frame}{Number Theory}{Division, Prime and Coprime}
%-------------------------------------------------------
\textbf{\large How to check whether a number $N$ is prime?}
\begin{algorithm}[H]
\begin{algorithmic}[1]
	\FOR{$i=2$ to $\sqrt{N}$}
		\IF{$i$ is divisor of $N$}
			\STATE \textbf{return} False
		\ENDIF
	\ENDFOR
\end{algorithmic}
\caption*{Pseudo-code for prime number checking.}
\end{algorithm}
\pause
~\\~\\
The time complexity of such algorithm is $O(\sqrt{N})$. \\
The space complexity of such algorithm is $O(1)$.

\end{frame}


%-------------------------------------------------------
\begin{frame}{Number Theory}{Division, Prime and Coprime}
%-------------------------------------------------------
\textbf{\large How to generate a prime number list from $1$ to $N$?}
\begin{enumerate}
	\item Check whether a number is prime one by one. \pause $O(N\sqrt{N})$ \pause
	\item Sieve of Eratosthenes 
\end{enumerate}
\end{frame}

%-------------------------------------------------------
\begin{frame}{Number Theory}{Division, Prime and Coprime}
%-------------------------------------------------------
\textbf{\large Sieve of Eratosthenes } \\
\begin{center}
\animategraphics[width=2in,height=2.2in,controls,loop]{10}{Images/output-}{0}{158}
\end{center}
\end{frame}


%-------------------------------------------------------
\begin{frame}{Number Theory}{Division, Prime and Coprime}
%-------------------------------------------------------
\textbf{\large How to generate a prime number list from $1$ to $N$?}
\begin{enumerate}
	\item Check whether a number is prime one by one. $O(N\sqrt{N})$
	\item Sieve of Eratosthenes \pause $O(N\log(N))$ \pause
	\item Linear Sieve
\end{enumerate}
\end{frame}


%-------------------------------------------------------
\begin{frame}{Number Theory}{Division, Prime and Coprime}
%-------------------------------------------------------
\begin{algorithm}[H]
\begin{algorithmic}[1]
	\FOR{$i=2$ to $N$}
		\STATE{set $IsPrime[i] = True$}
	\ENDFOR
	\FOR{$i=2$ to $N$}
		\IF{$IsPrime[i]$}
			\STATE{push $i$ into PrimeNumberList}
		\ENDIF
		\STATE{$j=0$}
		\WHILE{$j$ < CurrentPrimeNumber \textbf{and} $i \cdot $PrimeNumberList[$j$]$ < N$}
			\STATE{IsPrime[$i \cdot $PrimeNumberList[$j$]] = False}
			\IF{PrimeNumberList[$j$] is divisor of $i$}
				\STATE{\textbf{break}}
			\ENDIF
			\STATE{\textbf{increase} $j$}
		\ENDWHILE
	\ENDFOR
\end{algorithmic}
\caption*{Pseudo-code for linear sieve.}
\end{algorithm}
\end{frame}


%-------------------------------------------------------
\begin{frame}{Number Theory}{Division, Prime and Coprime}
%-------------------------------------------------------
\textbf{\large Linear Sieve} \\
The time complexity of such algorithm is $O(N)$. \\
The space complexity of such algorithm is $O(N)$. \\[0.5cm]
\pause
\textbf{Practice: POJ 2689} \\
Given $L, U(1\leq L < U < 2^{31}, U-L\leq 1,000,000)$, you are to find the two adjacent primes $C_1$ and $C_2 (L\leq C_1 < C_2 \leq U)$ that are closest (i.e. $C_2 - C_1$ is the minimum), and the two adjacent primes $D_1$ and $D_2 (L\leq D_1 < D_2 \leq U)$ that are the most distant (i.e. $D_2 - D_1$ is the maximum). 
\\~\\
\pause
Hint: You are definitely required to generate a prime number list in the range of $[L,U]$. But how?
\end{frame}


%-------------------------------------------------------
\begin{frame}{Optional: Mathematical Induction}
%-------------------------------------------------------
Mathematical induction is a mathematical proof technique used to prove a given statement about any well-ordered set. Most commonly, it is used to establish statements for the set of all natural numbers. \\~\\
\pause
It contains two steps:
\begin{enumerate}
\item \textbf{base case}, to prove the given statement for the least element in the well-ordered set. \pause
\item \textbf{inductive step}, to prove that, if the statement is assumed to be true for any element, then it must be true for the next element as well.
\end{enumerate}
\end{frame}

%-------------------------------------------------------
\begin{frame}{Optional: Mathematical Induction}
%-------------------------------------------------------
\textbf{\large An Example}
\\~\\
Prove $\sum_{i=0}^n = \frac{n(n+1)}{2}, \forall n\in \mathbb{N}$ \\[0.5cm]
\pause
\textbf{\sl Proof}~For the case $n=0$, we have $\sum_{i=0}^n=\frac{n(n+1)}{2}=0$. \\[0.2cm]
\pause
Assume that $\sum_{i=0}^n=\frac{n(n+1)}{2}$ when $n=k$, \\[0.2cm]
then $\sum_{i=0}^{k+1}=\frac{k(k+1)}{2}+k+1=\frac{k(k+1)+2(k+1)}{2}=\frac{(k+1)(k+2)}{2}$. \\[0.2cm]
Hence, the statement is true when $n=k+1$. \\[0.2cm]
\pause
Therefore, $\sum_{i=0}^n = \frac{n(n+1)}{2}, \forall n\in \mathbb{N}$. \#
\end{frame}


%-------------------------------------------------------
\begin{frame}{Number Theory}{Division, Prime and Coprime}
%-------------------------------------------------------
\textbf{\large Euclid Algorithm for Greatest Common Divisor(GCD)}
\begin{algorithm}[H]
\begin{algorithmic}[1]
	\STATE{\textbf{function} GCD(\textbf{integer} a,b)}
	\IF{b == 0}
		\STATE{\textbf{return} a}
	\ELSE
		\STATE{\textbf{return} GCD(b, a \textbf{mod} b)}
	\ENDIF
\end{algorithmic}
\caption*{Pseudo-code for linear sieve.}
\end{algorithm}
\end{frame}

%-------------------------------------------------------
\begin{frame}{Number Theory}{Division, Prime and Coprime}
%-------------------------------------------------------
\textbf{\large Euler's totient function}

\end{frame}


%-------------------------------------------------------
\subsection{Congruence Modulo}
%-------------------------------------------------------



%-------------------------------------------------------
\section{Introduction to Calculus}
\subsection{Differential of a Function}
%-------------------------------------------------------


%-------------------------------------------------------
\subsection{Calculus}
%-------------------------------------------------------




{\1
\begin{frame}[plain,noframenumbering]
  \finalpage{Thank you for attention! \\ Powereed by \LaTeX. }
\end{frame}}

\end{document}